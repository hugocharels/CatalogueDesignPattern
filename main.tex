
\documentclass[french]{article}

% Encodage et langue
\usepackage[utf8]{inputenc}
\usepackage[T1]{fontenc}
\usepackage[french]{babel}

% Mise en forme générale
\usepackage{fullpage}
\usepackage[parfill]{parskip}
\usepackage[hidelinks]{hyperref}  % Liens hypertext

% Graphiques et images
\usepackage{graphicx}
\usepackage{pgfplots}
\pgfplotsset{compat=1.18,legend to name={customlegend}}
\usepgfplotslibrary{statistics}
\usetikzlibrary{pgfplots.statistics}
\usepackage{tikz}
\usepackage{svg}
\usepackage{subcaption}
\usepackage{rotating}
\usepackage{float} % Pour positionner les images

% Mathématiques
\usepackage{amsmath}
\usepackage{amssymb}
\usepackage{amsfonts}
\usepackage{algorithm}
\usepackage{algpseudocode}

% Importation de code
\usepackage{listings}
\usepackage{listingsutf8}
\usepackage{xcolor}  % Couleur code

% Autres utilitaires
\usepackage{caption}
\usepackage{enumitem}
\usepackage{pifont}
\usepackage{multirow}
\usepackage{adjustbox}
\usepackage{lipsum} % Pour du texte factice
\usepackage{pdflscape}
\usepackage{adjustbox}
\usepackage{hyperref}
\usepackage[most]{tcolorbox}

% Définir le style par défaut
\tcbset{
  defaultstyle/.style={
    colback=gray!5!white,
    fonttitle=\bfseries
    
  }
}

% Pour le code
\lstset{
    language=Java,
    inputencoding=utf8/latin1,
    basicstyle=\ttfamily\small,
    keywordstyle=\color{blue},
    commentstyle=\color{green!70!black},
    stringstyle=\color{red},
    breaklines=true,
    showstringspaces=false,
    tabsize=4,
    frame=tb,
    numbers=left,
    numberstyle=\tiny\color{gray},
    captionpos=b
}

% Pour l'indentation
\usepackage{indentfirst} 
\setlength{\parindent}{1em} 

\usepackage{titlesec}
\titleformat{\section}[block]{\normalfont\Large\bfseries}{}{0em}{}


% Change font
%\usepackage[T1]{fontenc}
%\renewcommand*\rmdefault{}


\begin{document}

\begin{titlepage}
	\centering
    % \includegraphics[width=1\textwidth]{log.jpg}
    \vspace{2.5cm} 
    
    \noindent\rule{15cm}{0.4pt}
    \vspace{0.25cm} 
    {\huge\bfseries Catalogue de Design Pattern \par}
    \vspace{0.5cm}
    {\scshape\Large  \par}
    \vspace{0.5cm}
    \noindent\rule{15cm}{0.4pt}
    \vfill


% Bottom of the page
	{\Large\scshape Charels Hugo \par}
	\vspace{0.5cm}
	{\Large\scshape B-INFO\par}
	\vspace{1cm}
	{\large 2023-2024 \par}
\end{titlepage}


\tableofcontents
\clearpage

\newpage

\section{Patterns de Création (Creational)}


\subsection{Abstract Factory}

\subsubsection{Description}

Le design pattern Abstract Factory est un modèle de conception qui appartient à la catégorie des patrons de conception creational (de création). Son objectif principal est de fournir une interface pour créer des familles d'objets liés ou dépendants sans spécifier leurs classes concrètes. Cela signifie que le code client peut créer des objets sans avoir à connaître les détails spécifiques de leur implémentation.

Voici les principaux éléments qui composent le design pattern Abstract Factory :

\begin{enumerate}
    \item \textbf{Abstract Factory (Fabrique Abstraite)} :
    \begin{itemize}
        \item Il s'agit de l'interface définissant des méthodes pour créer chacun des types d'objets abstraits faisant partie de la famille d'objets.
        \item Chaque méthode de l'interface correspond à la création d'un type d'objet abstrait.
    \end{itemize}
    
    \item \textbf{Concrete Factory (Fabrique Concrète)} :
    \begin{itemize}
        \item Les implémentations concrètes de l'interface Abstract Factory.
        \item Chaque Concrete Factory est responsable de la création de toute une famille d'objets concrets.
    \end{itemize}
    
    \item \textbf{Abstract Product (Produit Abstrait)} :
    \begin{itemize}
        \item Interface ou classe abstraite définissant le comportement des objets de la famille.
        \item Chaque produit de la famille possède ses propres méthodes, mais ces méthodes sont déclarées dans l'interface ou la classe abstraite commune à tous les produits.
    \end{itemize}
    
    \item \textbf{Concrete Product (Produit Concret)} :
    \begin{itemize}
        \item Les implémentations concrètes des produits abstraits.
        \item Chaque Concrete Product est une implémentation spécifique d'un produit de la famille.
    \end{itemize}
    
    \item \textbf{Client} :
    \begin{itemize}
        \item Utilise l'interface de l'Abstract Factory pour créer des objets.
        \item N'a pas besoin de connaître les détails spécifiques de la création des objets.
        \item Travailler avec les objets via leurs interfaces abstraites.
    \end{itemize}
\end{enumerate}

Le processus de création d'objets avec le design pattern Abstract Factory se déroule comme suit :

\begin{enumerate}
    \item Le client appelle les méthodes de création de l'Abstract Factory pour obtenir des objets.
    \item L'Abstract Factory, en fonction de son type concret, crée les instances concrètes des produits de la famille.
    \item Le client utilise les objets créés via les interfaces abstraites, ce qui lui permet de rester indépendant des classes concrètes spécifiques.
\end{enumerate}

L'avantage principal de ce modèle est qu'il favorise la séparation des préoccupations en permettant aux clients de créer des familles d'objets apparentés sans avoir à connaître les détails de leur implémentation. Cela rend le code plus modulaire, plus flexible et plus facile à étendre. De plus, il favorise le principe de substitution de Liskov, car les objets peuvent être utilisés via leurs interfaces abstraites, permettant ainsi aux implémentations concrètes d'être interchangées facilement.

\subsubsection{Exemple}

Imaginons que nous construisons une application de rendu graphique qui doit fonctionner sur différentes plateformes, telles que Windows et Linux. Pour chaque plateforme, nous devons créer des boutons, des cases à cocher et des champs de texte avec un aspect spécifique à cette plateforme. Nous pouvons utiliser le Design Pattern Abstract Factory pour créer une fabrique abstraite qui sera implémentée par des fabriques concrètes pour chaque plateforme. Ainsi, notre code client peut créer des widgets sans se soucier de la plateforme sous-jacente.

\lstinputlisting[language=Java, caption=abstract\_factory.java]{code/creational/abstract_factory.java}



\newpage

\subsection{Builder}

\subsubsection{Description}

Le design pattern Builder est un modèle de conception appartenant à la catégorie des patrons de conception creational (de création). Son objectif principal est de séparer la construction d'un objet complexe de sa représentation, de sorte que le même processus de construction puisse créer différentes représentations.

Voici les principaux éléments qui composent le design pattern Builder :

\begin{enumerate}
    \item \textbf{Builder (Constructeur)} :
    \begin{itemize}
        \item Interface définissant les étapes de construction pour créer un objet complexe.
    \end{itemize}
    
    \item \textbf{Concrete Builder (Constructeur Concret)} :
    \begin{itemize}
        \item Implémente l'interface Builder pour fournir des étapes de construction concrètes pour un type spécifique d'objet complexe.
        \item Construit et assemble les parties de l'objet complexe selon les étapes spécifiées.
    \end{itemize}
    
    \item \textbf{Director (Directeur)} :
    \begin{itemize}
        \item Dirige le processus de construction en utilisant un Builder pour construire un objet complexe.
        \item Ne connaît pas les détails de la construction, mais utilise l'interface Builder pour orchestrer le processus.
    \end{itemize}
    
    \item \textbf{Product (Produit)} :
    \begin{itemize}
        \item Représente l'objet complexe en cours de construction.
        \item Peut être de n'importe quel type ou structure, en fonction de la logique de construction.
    \end{itemize}
\end{enumerate}

Le processus de construction d'un objet complexe avec le design pattern Builder se déroule comme suit :

\begin{enumerate}
    \item Le client crée un Builder concret et le passe au Directeur.
    \item Le Directeur utilise le Builder pour construire l'objet complexe en suivant les étapes définies.
    \item Une fois la construction terminée, le client récupère le produit du Builder.
\end{enumerate}

L'avantage principal de ce modèle est sa flexibilité et sa capacité à créer différentes représentations d'un même objet complexe. Il permet également de simplifier le code client en séparant la logique de construction de la logique métier. Cela facilite également l'ajout de nouvelles étapes de construction ou la modification de la logique de construction sans affecter le client.


\subsubsection{Exemple}

Supposons que nous construisons une application pour assembler des ordinateurs. Les ordinateurs peuvent avoir différentes configurations avec des composants variés, tels que le processeur, la carte graphique, la mémoire, etc. Nous pouvons utiliser le Design Pattern Builder pour définir une interface de construction abstraite et créer des constructeurs concrets pour chaque type d'ordinateur (gamer, bureautique, etc.). Ainsi, nous pouvons construire différents types d'ordinateurs en utilisant le même processus de construction.

\lstinputlisting[language=Java, caption=builder.java]{code/creational/builder.java}


\newpage

\subsection{Factory Method}

\subsubsection{Description}

Le design pattern Factory Method est un modèle de conception appartenant à la catégorie des patrons de conception creational (de création). Son objectif principal est de fournir une interface pour la création d'objets dans une classe, mais de permettre aux sous-classes de modifier le type d'objets qui seront instanciés.

Voici les principaux éléments qui composent le design pattern Factory Method :

\begin{enumerate}
    \item \textbf{Product (Produit)} :
    \begin{itemize}
        \item Interface ou classe abstraite définissant le type d'objets produits par le Factory Method.
    \end{itemize}
    
    \item \textbf{Concrete Product (Produit Concret)} :
    \begin{itemize}
        \item Implémentation concrète de l'interface Product.
        \item Chaque Concrete Product représente un type spécifique d'objet créé par le Factory Method.
    \end{itemize}
    
    \item \textbf{Creator (Créateur)} :
    \begin{itemize}
        \item Classe abstraite qui définit la méthode factoryMethod().
        \item Cette méthode est responsable de la création d'objets de type Product.
    \end{itemize}
    
    \item \textbf{Concrete Creator (Créateur Concret)} :
    \begin{itemize}
        \item Implémentation concrète de la classe Creator.
        \item Override la méthode factoryMethod() pour créer des instances spécifiques de Concrete Product.
    \end{itemize}
\end{enumerate}

Le processus de création d'objets avec le design pattern Factory Method se déroule comme suit :

\begin{enumerate}
    \item Le client appelle la méthode factoryMethod() de la classe Creator pour obtenir une instance de Product.
    \item La classe Creator, qui peut être une classe abstraite ou une classe concrète, crée et retourne une instance de Concrete Product en appelant la méthode factoryMethod().
    \item Le client utilise ensuite l'objet Product obtenu via l'interface commune, sans avoir besoin de connaître la classe concrète réelle de l'objet.
\end{enumerate}

L'avantage principal de ce modèle est qu'il permet de déléguer la responsabilité de la création d'objets à des sous-classes, ce qui permet une meilleure extensibilité et une réduction du couplage entre les classes. Il facilite également l'ajout de nouveaux types d'objets sans avoir à modifier le code existant.


\subsubsection{Exemple}

Supposons que nous développons un logiciel de traitement d'images avec différents types de filtres (filtre noir et blanc, filtre sepia, etc.). Nous pouvons utiliser le Design Pattern Factory Method en définissant une classe abstraite "Filter" avec une méthode abstraite "apply", qui sera implémentée par les sous-classes pour créer des filtres spécifiques.

\lstinputlisting[language=Java, caption=factory\_method.java]{code/creational/factory_method.java}


\newpage

\subsection{Prototype}

\subsubsection{Description}

Le design pattern Prototype est un modèle de conception appartenant à la catégorie des patrons de conception creational (de création). Son objectif principal est de permettre la création d'objets en clonant une instance existante plutôt qu'en les instanciant à partir de zéro.

Voici les principaux éléments qui composent le design pattern Prototype :

\begin{enumerate}
    \item \textbf{Prototype} :
    \begin{itemize}
        \item Interface ou classe abstraite définissant la méthode clone().
        \item Cette méthode est utilisée pour créer une copie profonde ou superficielle de l'objet, selon les besoins.
    \end{itemize}
    
    \item \textbf{Concrete Prototype (Prototype Concret)} :
    \begin{itemize}
        \item Implémentation concrète de l'interface Prototype.
        \item Définit la logique de clonage de l'objet.
    \end{itemize}
    
    \item \textbf{Client} :
    \begin{itemize}
        \item Utilise le Prototype pour créer de nouveaux objets en les clonant.
        \item Ne nécessite pas de connaître les détails de l'implémentation du clonage.
    \end{itemize}
\end{enumerate}

Le processus de création d'objets avec le design pattern Prototype se déroule comme suit :

\begin{enumerate}
    \item Le client demande la création d'un nouvel objet en utilisant un objet Prototype existant.
    \item Le Prototype, qui peut être une classe abstraite ou une classe concrète, utilise sa méthode clone() pour créer une copie de lui-même.
    \item Le client utilise ensuite l'objet cloné selon ses besoins.
\end{enumerate}

L'avantage principal de ce modèle est qu'il permet de créer de nouveaux objets avec un minimum d'effort, en évitant le processus de création coûteux. Il permet également de réduire la duplication de code et d'offrir une meilleure flexibilité en permettant la création de nouveaux types d'objets en utilisant des prototypes existants.

\subsubsection{Exemple}

Supposons que nous développons une application de dessin où les utilisateurs peuvent créer des formes géométriques. Pour créer une nouvelle forme, nous pouvons utiliser le Design Pattern Prototype en définissant une interface "Shape" avec une méthode "clone" qui sera implémentée par les sous-classes pour copier l'objet existant.

\lstinputlisting[language=Java, caption=prototype.java]{code/creational/prototype.java}


\newpage

\subsection{Singleton}

\subsubsection{Description}

Le design pattern Singleton est un modèle de conception appartenant à la catégorie des patrons de conception creational (de création). Son objectif principal est de garantir qu'une classe n'a qu'une seule instance et de fournir un point d'accès global à cette instance.

Voici les principaux éléments qui composent le design pattern Singleton :

\begin{enumerate}
    \item \textbf{Singleton} :
    \begin{itemize}
        \item Classe avec une méthode statique qui retourne toujours la même instance de cette classe.
        \item Le constructeur de la classe est généralement rendu privé pour empêcher l'instanciation directe de la classe en dehors de la classe elle-même.
    \end{itemize}
\end{enumerate}

Le processus d'utilisation du design pattern Singleton est assez simple :

\begin{enumerate}
    \item Les clients accèdent à l'instance unique de la classe Singleton en appelant la méthode statique de la classe.
    \item Si l'instance n'existe pas encore, elle est créée et stockée dans un champ statique privé de la classe.
    \item L'instance unique est ensuite retournée à chaque appel de la méthode statique.
\end{enumerate}

L'avantage principal de ce modèle est qu'il garantit qu'une classe n'a qu'une seule instance dans l'ensemble du programme, ce qui peut être utile pour des ressources partagées telles que des bases de données ou des fichiers de configuration. Cela évite également le gaspillage de ressources en évitant la création répétée d'instances et offre un point d'accès global pour accéder à cette instance unique.

Cependant, l'utilisation abusive du Singleton peut conduire à des problèmes de test unitaire et à des dépendances cachées, il convient donc de l'utiliser avec discernement.


\subsubsection{Exemple}

Supposons que nous développons une application qui a besoin d'une classe "Configuration" pour stocker les paramètres de configuration de l'application. Nous pouvons utiliser le Design Pattern Singleton pour s'assurer qu'il n'y a qu'une seule instance de la classe "Configuration" qui est partagée par l'ensemble de l'application.

\lstinputlisting[language=Java, caption=singleton.java]{code/creational/singleton.java}

\newpage

% --------------------------------------------------------------------------------



\section{Patterns de Structures (Structural)}

\subsection{Adapter}


\subsubsection{Description}

Le design pattern Adapter est un modèle de conception appartenant à la catégorie des patrons de conception structurels. Son objectif principal est de permettre à des interfaces incompatibles de travailler ensemble en convertissant l'interface d'une classe en une autre interface attendue par le client.

Voici les principaux éléments qui composent le design pattern Adapter :

\begin{enumerate}
    \item \textbf{Target (Cible)} :
    \begin{itemize}
        \item Interface que le client utilise pour interagir avec le système.
    \end{itemize}
    
    \item \textbf{Adapter (Adaptateur)} :
    \begin{itemize}
        \item Classe qui adapte l'interface d'une classe existante (Adaptee) à l'interface Target attendue par le client.
        \item Implémente l'interface Target et contient une instance de la classe Adaptee.
    \end{itemize}
    
    \item \textbf{Adaptee (Adapté)} :
    \begin{itemize}
        \item Classe existante dont l'interface n'est pas compatible avec l'interface Target.
        \item La classe que l'Adapter va adapter pour qu'elle puisse être utilisée par le client.
    \end{itemize}
\end{enumerate}

Le processus d'utilisation du design pattern Adapter se déroule comme suit :

\begin{enumerate}
    \item Le client utilise l'interface Target pour interagir avec le système.
    \item L'Adapter reçoit les appels de l'interface Target et les convertit en appels appropriés à l'interface de l'Adaptee.
    \item L'Adaptee exécute les opérations demandées et retourne les résultats à l'Adapter.
    \item L'Adapter convertit ensuite les résultats de l'Adaptee en un format compatible avec l'interface Target et les renvoie au client.
\end{enumerate}

L'avantage principal de ce modèle est qu'il permet d'intégrer des classes existantes dans de nouveaux systèmes sans avoir à modifier leur code source. Cela favorise la réutilisabilité du code et permet d'ajouter de nouvelles fonctionnalités sans affecter les composants existants. Cependant, l'utilisation abusive de ce modèle peut entraîner une complexité accrue du code en raison de l'ajout de plusieurs couches d'adaptation.


\subsubsection{Exemple}

Supposons que nous avons une classe "LegacyPrinter" qui utilise une ancienne interface pour l'impression de documents. Nous développons une nouvelle classe "ModernPrinter" qui utilise une interface différente pour l'impression. Pour que notre code client puisse utiliser les deux types d'imprimantes de manière interchangeable, nous pouvons utiliser le Design Pattern Adapter pour créer un adaptateur qui convertit l'interface de "ModernPrinter" en celle de "LegacyPrinter".

\lstinputlisting[language=Java, caption=adapter.java]{code/structural/adapter.java}


\newpage

\subsection{Bridge}

\subsubsection{Description}

Le design pattern Bridge est un modèle de conception appartenant à la catégorie des patrons de conception structurels. Son objectif principal est de séparer l'abstraction d'une classe de son implémentation, permettant ainsi à ces deux parties de varier indépendamment.

Voici les principaux éléments qui composent le design pattern Bridge :

\begin{enumerate}
    \item \textbf{Abstraction} :
    \begin{itemize}
        \item Interface qui définit les méthodes abstraites utilisées par le client.
        \item Contient une référence à un objet Implementor.
    \end{itemize}
    
    \item \textbf{Refined Abstraction (Abstraction Affinée)} :
    \begin{itemize}
        \item Implémentation spécifique de l'interface Abstraction.
        \item Peut ajouter des fonctionnalités supplémentaires.
    \end{itemize}
    
    \item \textbf{Implementor (Implémenteur)} :
    \begin{itemize}
        \item Interface qui définit les méthodes abstraites utilisées par Abstraction.
    \end{itemize}
    
    \item \textbf{Concrete Implementor (Implémenteur Concret)} :
    \begin{itemize}
        \item Implémentation concrète de l'interface Implementor.
        \item Contient la logique détaillée de l'implémentation.
    \end{itemize}
\end{enumerate}

Le processus d'utilisation du design pattern Bridge se déroule comme suit :

\begin{enumerate}
    \item Le client utilise l'interface Abstraction pour interagir avec le système.
    \item L'Abstraction délègue une partie de son implémentation à l'objet Implementor référencé.
    \item Les différentes implémentations de Implementor peuvent être échangées dynamiquement sans affecter Abstraction.
\end{enumerate}

L'avantage principal de ce modèle est qu'il permet de séparer complètement l'abstraction de son implémentation, ce qui facilite l'évolution et la maintenance du code. Il permet également de réduire le couplage entre les classes en les reliant par des interfaces plutôt que par des implémentations concrètes. Cependant, cela peut introduire une complexité supplémentaire dans le code en raison de l'ajout de plusieurs couches d'abstraction et d'implémentation.


\subsubsection{Exemple}

Supposons que nous développons un système de formes géométriques avec différents types de dessin (par exemple, dessin vectoriel et dessin en raster). Au lieu de créer une classe pour chaque combinaison de forme et de dessin, nous pouvons utiliser le Design Pattern Bridge pour diviser la hiérarchie en deux parties : l'abstraction (Forme) et l'implémentation (Dessin). Ainsi, nous pouvons créer des ponts (bridges) entre les formes et les dessins pour obtenir différentes combinaisons de formes et de dessins.

\lstinputlisting[language=Java, caption=bridge.java]{code/structural/bridge.java}


\newpage

\subsection{Composite}

\subsubsection{Description}


Le design pattern Composite est un modèle de conception appartenant à la catégorie des patrons de conception structurels. Son objectif principal est de permettre la composition d'objets en structures arborescentes pour représenter les hiérarchies partie-tout.

Voici les principaux éléments qui composent le design pattern Composite :

\begin{enumerate}
    \item \textbf{Component (Composant)} :
    \begin{itemize}
        \item Interface ou classe abstraite commune à tous les composants de la structure.
        \item Définit les méthodes communes pour manipuler les composants.
    \end{itemize}
    
    \item \textbf{Leaf (Feuille)} :
    \begin{itemize}
        \item Implémentation concrète de la classe Component.
        \item Représente les éléments individuels de la structure qui n'ont pas de sous-composants.
    \end{itemize}
    
    \item \textbf{Composite} :
    \begin{itemize}
        \item Implémentation concrète de la classe Component.
        \item Représente les éléments de la structure qui ont des sous-composants.
        \item Contient une liste de références vers ses sous-composants.
    \end{itemize}
\end{enumerate}

Le processus d'utilisation du design pattern Composite se déroule comme suit :

\begin{enumerate}
    \item Les clients manipulent les composants de la structure via l'interface commune Component.
    \item Les opérations sont propagées récursivement dans la structure, avec chaque composant (feuille ou composite) déléguant les appels à ses sous-composants le cas échéant.
    \item Cela permet de traiter uniformément les composants individuels et les structures complexes de manière transparente.
\end{enumerate}

L'avantage principal de ce modèle est qu'il permet de manipuler des structures arborescentes de manière uniforme, en traitant les composants individuels et les structures composites de la même manière. Cela simplifie le code client en permettant de traiter une structure complexe comme une unité unique, tout en offrant une grande flexibilité dans la manipulation des composants. Cependant, cela peut rendre certaines opérations plus complexes en raison de la nécessité de gérer la récursion.


\subsubsection{Exemple}

Supposons que nous développons une application de modélisation de formes graphiques. Nous avons des formes simples (cercle, carré) et des groupes de formes qui peuvent contenir d'autres formes, y compris d'autres groupes. Avec le Design Pattern Composite, nous pouvons traiter les formes individuelles et les groupes de formes de manière homogène, ce qui facilite les opérations de dessin, de déplacement, etc.

\lstinputlisting[language=Java, caption=composite.java]{code/structural/composite.java}


\newpage

\subsection{Decorator}

\subsubsection{Description}

Le design pattern Decorator est un modèle de conception appartenant à la catégorie des patrons de conception structurels. Son objectif principal est d'ajouter dynamiquement de nouvelles fonctionnalités à un objet en les enveloppant dans des objets décorateurs plutôt que de les étendre par héritage.

Voici les principaux éléments qui composent le design pattern Decorator :

\begin{enumerate}
    \item \textbf{Component (Composant)} :
    \begin{itemize}
        \item Interface ou classe abstraite commune à tous les composants à décorer.
        \item Définit les méthodes de base que les décorateurs et les composants concrets implémentent.
    \end{itemize}
    
    \item \textbf{Concrete Component (Composant Concret)} :
    \begin{itemize}
        \item Implémentation concrète de la classe Component.
        \item Représente l'objet de base auquel des fonctionnalités supplémentaires peuvent être ajoutées.
    \end{itemize}
    
    \item \textbf{Decorator (Décorateur)} :
    \begin{itemize}
        \item Classe abstraite qui étend Component et enveloppe les composants concrets.
        \item Contient une référence à un objet de type Component.
    \end{itemize}
    
    \item \textbf{Concrete Decorator (Décorateur Concret)} :
    \begin{itemize}
        \item Implémentation concrète de la classe Decorator.
        \item Ajoute des fonctionnalités supplémentaires à l'objet de base en le décorant.
    \end{itemize}
\end{enumerate}

Le processus d'utilisation du design pattern Decorator se déroule comme suit :

\begin{enumerate}
    \item Les clients manipulent les composants à travers l'interface Component.
    \item Les fonctionnalités supplémentaires sont ajoutées dynamiquement en enveloppant le composant de base dans des décorateurs appropriés.
    \item Les décorateurs peuvent être empilés pour ajouter plusieurs fonctionnalités en cascade.
    \item Chaque décorateur transmet les appels aux méthodes de base du composant au composant sous-jacent, permettant ainsi une chaîne de responsabilité.
\end{enumerate}

L'avantage principal de ce modèle est qu'il permet d'ajouter des fonctionnalités supplémentaires à un objet de manière flexible et dynamique, sans avoir à modifier sa structure de classe. Cela favorise la réutilisabilité du code en permettant la combinaison libre de fonctionnalités à la volée. Cependant, cela peut rendre la lecture du code plus complexe en raison de la présence de plusieurs couches de décorateurs.


\subsubsection{Exemple}

Supposons que nous développons une application de café et nous avons différentes boissons (expresso, café au lait) avec des options supplémentaires (lait, chocolat, sucre). Avec le Design Pattern Decorator, nous pouvons ajouter les options supplémentaires de manière dynamique à chaque boisson sans avoir besoin de créer de nombreuses classes pour chaque combinaison possible.

\lstinputlisting[language=Java, caption=decorator.java]{code/structural/decorator.java}


\newpage

\subsection{Facade}

\subsubsection{Description}

Le design pattern Facade est un modèle de conception appartenant à la catégorie des patrons de conception structurels. Son objectif principal est de fournir une interface unifiée à un ensemble d'interfaces d'un sous-système afin de simplifier son utilisation et de masquer sa complexité.

Voici les principaux éléments qui composent le design pattern Facade :

\begin{enumerate}
    \item \textbf{Facade (Facade)} :
    \begin{itemize}
        \item Classe qui fournit une interface unifiée à un ensemble d'interfaces plus complexes dans un sous-système.
        \item Cache les détails de l'implémentation et simplifie l'interaction avec le sous-système.
    \end{itemize}
    
    \item \textbf{Subsystem (Sous-système)} :
    \begin{itemize}
        \item Ensemble de classes et d'interfaces qui implémentent les fonctionnalités du système.
        \item Ces classes ne sont pas directement accessibles par les clients mais sont utilisées par la facade pour réaliser les fonctionnalités demandées.
    \end{itemize}
\end{enumerate}

Le processus d'utilisation du design pattern Facade se déroule comme suit :

\begin{enumerate}
    \item Les clients interagissent avec la Facade pour accéder aux fonctionnalités du sous-système.
    \item La Facade traduit ces demandes en appels appropriés aux classes et interfaces du sous-système.
    \item La Facade simplifie ainsi l'utilisation du sous-système en fournissant une interface plus conviviale et en masquant sa complexité interne.
\end{enumerate}

L'avantage principal de ce modèle est qu'il permet de simplifier l'utilisation d'un sous-système complexe en fournissant une interface unifiée et conviviale. Cela permet aux clients de ne pas avoir à connaître les détails de l'implémentation du sous-système, ce qui favorise la modularité et la maintenance du code. Cependant, cela peut également limiter la flexibilité en cachant les détails d'implémentation, et il est important de concevoir soigneusement les interfaces de la Facade pour répondre aux besoins des clients.


\subsubsection{Exemple}

Supposons que nous développons une application pour gérer une voiture avec plusieurs sous-systèmes tels que le moteur, les freins, l'électronique, etc. Au lieu d'interagir directement avec chaque sous-système, nous pouvons créer une Facade Car pour regrouper les fonctionnalités de chaque sous-système et fournir une interface unique pour interagir avec la voiture.

\lstinputlisting[language=Java, caption=facade.java]{code/structural/facade.java}


\newpage

\subsection{Flyweight}

\subsubsection{Description}

Le design pattern Flyweight est un modèle de conception appartenant à la catégorie des patrons de conception structurels. Son objectif principal est de minimiser l'utilisation de la mémoire ou du calcul en partageant autant que possible les données similaires entre plusieurs objets.

Voici les principaux éléments qui composent le design pattern Flyweight :

\begin{enumerate}
    \item \textbf{Flyweight (Poids Léger)} :
    \begin{itemize}
        \item Interface ou classe abstraite commune à tous les objets légers.
        \item Contient les méthodes partagées entre les objets légers.
    \end{itemize}
    
    \item \textbf{ConcreteFlyweight (Poids Léger Concret)} :
    \begin{itemize}
        \item Implémentation concrète de l'interface Flyweight.
        \item Représente un objet léger partagé qui stocke l'état intrinsèque (partagé) et ne dépend pas du contexte externe.
    \end{itemize}
    
    \item \textbf{UnsharedConcreteFlyweight (Poids Léger Non Partagé)} :
    \begin{itemize}
        \item Implémentation concrète de l'interface Flyweight.
        \item Représente un objet léger qui ne peut pas être partagé et stocke l'état extrinsèque (non partagé).
    \end{itemize}
    
    \item \textbf{FlyweightFactory (Fabrique de Poids Légers)} :
    \begin{itemize}
        \item Gère et fournit les objets légers existants.
        \item Assure le partage des objets légers lors de leur création.
    \end{itemize}
\end{enumerate}

Le processus d'utilisation du design pattern Flyweight se déroule comme suit :

\begin{enumerate}
    \item Les clients demandent des objets légers à la FlyweightFactory.
    \item Si l'objet léger existe déjà dans la FlyweightFactory, il est renvoyé au client.
    \item Sinon, un nouvel objet léger est créé et stocké dans la FlyweightFactory pour une utilisation ultérieure.
    \item Les clients manipulent les objets légers via l'interface Flyweight.
\end{enumerate}

L'avantage principal de ce modèle est qu'il permet de réduire la consommation de mémoire en partageant les objets similaires entre plusieurs instances. Cela peut être particulièrement utile lorsque de nombreux objets doivent être créés, mais leur état est souvent répétitif. Cependant, cela peut également rendre le code plus complexe en raison de la nécessité de gérer les états partagés et non partagés.


\subsubsection{Exemple}

Supposons que nous développons un éditeur de texte où chaque caractère est représenté par un objet Character. Plutôt que de créer un nouvel objet Character pour chaque caractère du texte, nous pouvons utiliser le Design Pattern Flyweight pour stocker les caractères déjà créés dans un cache et les réutiliser lorsque le même caractère est demandé à nouveau.

\lstinputlisting[language=Java, caption=flyweight.java]{code/structural/flyweight.java}


\newpage

\subsection{Proxy}

\subsubsection{Description}

Le design pattern Proxy est un modèle de conception appartenant à la catégorie des patrons de conception structurels. Son objectif principal est de fournir un substitut ou un espace réservé à un autre objet afin de contrôler l'accès à celui-ci ou de fournir des fonctionnalités supplémentaires.

Voici les principaux éléments qui composent le design pattern Proxy :

\begin{enumerate}
    \item \textbf{Subject (Sujet)} :
    \begin{itemize}
        \item Interface ou classe abstraite commune à l'objet réel et à son proxy.
        \item Définit les méthodes communes que le proxy et l'objet réel implémentent.
    \end{itemize}
    
    \item \textbf{RealSubject (Sujet Réel)} :
    \begin{itemize}
        \item Implémentation concrète de l'interface Subject.
        \item Représente l'objet réel dont l'accès est contrôlé par le proxy.
    \end{itemize}
    
    \item \textbf{Proxy} :
    \begin{itemize}
        \item Implémentation concrète de l'interface Subject.
        \item Contrôle l'accès à l'objet réel et fournit des fonctionnalités supplémentaires si nécessaire.
        \item Peut retarder la création et l'initialisation de l'objet réel jusqu'à ce qu'il soit vraiment nécessaire (lazy initialization).
    \end{itemize}
\end{enumerate}

Le processus d'utilisation du design pattern Proxy se déroule comme suit :

\begin{enumerate}
    \item Les clients interagissent avec le proxy à travers l'interface Subject.
    \item Si l'objet réel n'est pas encore créé, le proxy peut le créer et l'initialiser de manière transparente.
    \item Le proxy transmet ensuite les appels à l'objet réel ou effectue des traitements supplémentaires avant ou après la transmission de l'appel.
\end{enumerate}

L'avantage principal de ce modèle est qu'il permet de contrôler l'accès à l'objet réel et de fournir des fonctionnalités supplémentaires sans modifier son code. Cela peut être utile pour ajouter des fonctionnalités telles que la mise en cache, la journalisation, la sécurité ou la gestion des ressources. Cependant, cela peut également introduire une complexité supplémentaire dans le code en raison de la présence de plusieurs couches de proxy.


\subsubsection{Exemple}

Supposons que nous développons une application pour télécharger des fichiers depuis Internet. Le téléchargement des fichiers peut être coûteux en temps, nous pouvons donc utiliser un Proxy pour vérifier les autorisations de l'utilisateur avant de permettre le téléchargement et pour mettre en cache les fichiers téléchargés pour une utilisation ultérieure.

\lstinputlisting[language=Java, caption=proxy.java]{code/structural/proxy.java}


\newpage

\section{Patterns de Comportement (Behavioral)}

\subsection{Chain of Responsibility}

\subsubsection{Description}



\subsubsection{Exemple}



\subsection{Command}

\subsubsection{Description}



\subsubsection{Exemple}



\subsection{Interpreter}

\subsubsection{Description}



\subsubsection{Exemple}



\subsection{Iterator}

\subsubsection{Description}



\subsubsection{Exemple}



\subsection{Mediator}

\subsubsection{Description}



\subsubsection{Exemple}



\subsection{Memento}

\subsubsection{Description}



\subsubsection{Exemple}



\subsection{Observer}

\subsubsection{Description}



\subsubsection{Exemple}



\subsection{State}

\subsubsection{Description}



\subsubsection{Exemple}



\subsection{Strategy}

\subsubsection{Description}



\subsubsection{Exemple}



\subsection{Template Method}

\subsubsection{Description}



\subsubsection{Exemple}



\subsection{Visitor}

\subsubsection{Description}



\subsubsection{Exemple}




\end{document}
